% Options for packages loaded elsewhere
\PassOptionsToPackage{unicode}{hyperref}
\PassOptionsToPackage{hyphens}{url}
%
\documentclass[
]{article}
\title{WhatsInSeason}
\author{Annie Johansson}
\date{5/26/2022}

\usepackage{amsmath,amssymb}
\usepackage{lmodern}
\usepackage{iftex}
\ifPDFTeX
  \usepackage[T1]{fontenc}
  \usepackage[utf8]{inputenc}
  \usepackage{textcomp} % provide euro and other symbols
\else % if luatex or xetex
  \usepackage{unicode-math}
  \defaultfontfeatures{Scale=MatchLowercase}
  \defaultfontfeatures[\rmfamily]{Ligatures=TeX,Scale=1}
\fi
% Use upquote if available, for straight quotes in verbatim environments
\IfFileExists{upquote.sty}{\usepackage{upquote}}{}
\IfFileExists{microtype.sty}{% use microtype if available
  \usepackage[]{microtype}
  \UseMicrotypeSet[protrusion]{basicmath} % disable protrusion for tt fonts
}{}
\makeatletter
\@ifundefined{KOMAClassName}{% if non-KOMA class
  \IfFileExists{parskip.sty}{%
    \usepackage{parskip}
  }{% else
    \setlength{\parindent}{0pt}
    \setlength{\parskip}{6pt plus 2pt minus 1pt}}
}{% if KOMA class
  \KOMAoptions{parskip=half}}
\makeatother
\usepackage{xcolor}
\IfFileExists{xurl.sty}{\usepackage{xurl}}{} % add URL line breaks if available
\IfFileExists{bookmark.sty}{\usepackage{bookmark}}{\usepackage{hyperref}}
\hypersetup{
  pdftitle={WhatsInSeason},
  pdfauthor={Annie Johansson},
  hidelinks,
  pdfcreator={LaTeX via pandoc}}
\urlstyle{same} % disable monospaced font for URLs
\usepackage[margin=1in]{geometry}
\usepackage{color}
\usepackage{fancyvrb}
\newcommand{\VerbBar}{|}
\newcommand{\VERB}{\Verb[commandchars=\\\{\}]}
\DefineVerbatimEnvironment{Highlighting}{Verbatim}{commandchars=\\\{\}}
% Add ',fontsize=\small' for more characters per line
\usepackage{framed}
\definecolor{shadecolor}{RGB}{248,248,248}
\newenvironment{Shaded}{\begin{snugshade}}{\end{snugshade}}
\newcommand{\AlertTok}[1]{\textcolor[rgb]{0.94,0.16,0.16}{#1}}
\newcommand{\AnnotationTok}[1]{\textcolor[rgb]{0.56,0.35,0.01}{\textbf{\textit{#1}}}}
\newcommand{\AttributeTok}[1]{\textcolor[rgb]{0.77,0.63,0.00}{#1}}
\newcommand{\BaseNTok}[1]{\textcolor[rgb]{0.00,0.00,0.81}{#1}}
\newcommand{\BuiltInTok}[1]{#1}
\newcommand{\CharTok}[1]{\textcolor[rgb]{0.31,0.60,0.02}{#1}}
\newcommand{\CommentTok}[1]{\textcolor[rgb]{0.56,0.35,0.01}{\textit{#1}}}
\newcommand{\CommentVarTok}[1]{\textcolor[rgb]{0.56,0.35,0.01}{\textbf{\textit{#1}}}}
\newcommand{\ConstantTok}[1]{\textcolor[rgb]{0.00,0.00,0.00}{#1}}
\newcommand{\ControlFlowTok}[1]{\textcolor[rgb]{0.13,0.29,0.53}{\textbf{#1}}}
\newcommand{\DataTypeTok}[1]{\textcolor[rgb]{0.13,0.29,0.53}{#1}}
\newcommand{\DecValTok}[1]{\textcolor[rgb]{0.00,0.00,0.81}{#1}}
\newcommand{\DocumentationTok}[1]{\textcolor[rgb]{0.56,0.35,0.01}{\textbf{\textit{#1}}}}
\newcommand{\ErrorTok}[1]{\textcolor[rgb]{0.64,0.00,0.00}{\textbf{#1}}}
\newcommand{\ExtensionTok}[1]{#1}
\newcommand{\FloatTok}[1]{\textcolor[rgb]{0.00,0.00,0.81}{#1}}
\newcommand{\FunctionTok}[1]{\textcolor[rgb]{0.00,0.00,0.00}{#1}}
\newcommand{\ImportTok}[1]{#1}
\newcommand{\InformationTok}[1]{\textcolor[rgb]{0.56,0.35,0.01}{\textbf{\textit{#1}}}}
\newcommand{\KeywordTok}[1]{\textcolor[rgb]{0.13,0.29,0.53}{\textbf{#1}}}
\newcommand{\NormalTok}[1]{#1}
\newcommand{\OperatorTok}[1]{\textcolor[rgb]{0.81,0.36,0.00}{\textbf{#1}}}
\newcommand{\OtherTok}[1]{\textcolor[rgb]{0.56,0.35,0.01}{#1}}
\newcommand{\PreprocessorTok}[1]{\textcolor[rgb]{0.56,0.35,0.01}{\textit{#1}}}
\newcommand{\RegionMarkerTok}[1]{#1}
\newcommand{\SpecialCharTok}[1]{\textcolor[rgb]{0.00,0.00,0.00}{#1}}
\newcommand{\SpecialStringTok}[1]{\textcolor[rgb]{0.31,0.60,0.02}{#1}}
\newcommand{\StringTok}[1]{\textcolor[rgb]{0.31,0.60,0.02}{#1}}
\newcommand{\VariableTok}[1]{\textcolor[rgb]{0.00,0.00,0.00}{#1}}
\newcommand{\VerbatimStringTok}[1]{\textcolor[rgb]{0.31,0.60,0.02}{#1}}
\newcommand{\WarningTok}[1]{\textcolor[rgb]{0.56,0.35,0.01}{\textbf{\textit{#1}}}}
\usepackage{graphicx}
\makeatletter
\def\maxwidth{\ifdim\Gin@nat@width>\linewidth\linewidth\else\Gin@nat@width\fi}
\def\maxheight{\ifdim\Gin@nat@height>\textheight\textheight\else\Gin@nat@height\fi}
\makeatother
% Scale images if necessary, so that they will not overflow the page
% margins by default, and it is still possible to overwrite the defaults
% using explicit options in \includegraphics[width, height, ...]{}
\setkeys{Gin}{width=\maxwidth,height=\maxheight,keepaspectratio}
% Set default figure placement to htbp
\makeatletter
\def\fps@figure{htbp}
\makeatother
\setlength{\emergencystretch}{3em} % prevent overfull lines
\providecommand{\tightlist}{%
  \setlength{\itemsep}{0pt}\setlength{\parskip}{0pt}}
\setcounter{secnumdepth}{-\maxdimen} % remove section numbering
\usepackage{amsmath}
\usepackage{booktabs}
\usepackage{caption}
\usepackage{longtable}
\ifLuaTeX
  \usepackage{selnolig}  % disable illegal ligatures
\fi

\begin{document}
\maketitle

WhatsInSeason is an r package intended to help you eat more consciously.
It can retrieve and visualize nutritional and climate-impact data about
most types of fresh produce.

\hypertarget{water-footprint-data}{%
\section{Water Footprint data}\label{water-footprint-data}}

Over 2.7 billion people are affected by scarce water resources yearly.
The water footprint metric, created by Arjen Hoekstra, is a type of
environmental footprint that helps us to understand how the human
consumption of different products is affecting the earth's natural water
sources. Making conscious choices about which products to consume will
aid in reducing our water foot print and ultimately preserve water for
people and nature.

The water footprint data used in this package comes from The Water
Footprint Network (\url{https://waterfootprint.org/en/}). The dataset
comprises the green, blue, and grey water footprints for a wide range of
food products on a global, national, and regional scale.

\hypertarget{types-of-water-footprints}{%
\paragraph{Types of Water Footprints}\label{types-of-water-footprints}}

\captionsetup[table]{labelformat=empty,skip=1pt}
\begin{longtable}{ll}
\toprule
Type & Explanation \\ 
\midrule
Green & Water acquired by the crop/plant directly from natural sources, such as precipitation. \\ 
Blue & Water that has been sourced from surface or groundwater resources. \\ 
Grey & The amount of freshwater that needs to be discharged into the water source to remove pollutants and meet water quality standards. \\ 
\bottomrule
\end{longtable}

Read more:
\url{https://waterfootprint.org/en/water-footprint/what-is-water-footprint/}

\hypertarget{water-footprint-functions}{%
\subsection{Water Footprint Functions}\label{water-footprint-functions}}

\hypertarget{retrieve-and-visualize-global-water-footprint-data}{%
\subsubsection{Retrieve and visualize global water footprint
data}\label{retrieve-and-visualize-global-water-footprint-data}}

The function \textbf{wf\_global} creates a table of water footprint data
for a specific food item.

\begin{Shaded}
\begin{Highlighting}[]
\FunctionTok{wf\_global}\NormalTok{(banana)}
\end{Highlighting}
\end{Shaded}

\begin{verbatim}
## # A tibble: 1 x 7
##   format width height colorspace matte filesize density
##   <chr>  <int>  <int> <chr>      <lgl>    <int> <chr>  
## 1 JPEG     300    300 sRGB       FALSE    20142 180x180
\end{verbatim}

\begin{verbatim}
## $`water footprint`
##                                         Product WF_Type AverageWF
## 691 Bananas including plantains, fresh or dried   Green       660
## 692 Bananas including plantains, fresh or dried    Blue        97
## 693 Bananas including plantains, fresh or dried    Grey        33
## 
## $nutrition
##                           search_result calories percent_nutrients  fat
## 1                                Banana       89              1.09 0.33
## 2 Bananas, Dehydrated, or Banana Powder      346              3.89 1.81
## 3                          Bananas, Raw       89              1.09 0.33
## 4                         Banana Pepper       27              1.66 0.45
##   carbohydrates fibre
## 1         22.84   2.6
## 2         88.28   9.9
## 3         22.84   2.6
## 4          5.35   3.4
\end{verbatim}

You can plot this data with the function \textbf{wf\_global\_plot}:

\begin{Shaded}
\begin{Highlighting}[]
\FunctionTok{plot\_wf\_global}\NormalTok{(banana)}
\end{Highlighting}
\end{Shaded}

\includegraphics{Package-Vignette_files/figure-latex/unnamed-chunk-3-1.pdf}

\hypertarget{nutritional-data}{%
\section{Nutritional data}\label{nutritional-data}}

Several functions in the WhatsInSeason package allow you to retrieve and
visualise nutritional data of fresh produce items. This data is taken
from the Food Database API \url{https://www.edamam.com}. With these
functions you can:

\begin{itemize}
\tightlist
\item
  Search for food items in the database.\\
\item
  View data about nutrients, calories, carbohydrates, fat, and fibre of
  food items.\\
\item
  Compare two different food items based on a nutritional metric.\\
\item
  Plot the nutrional data of the food item.
\end{itemize}

\end{document}
